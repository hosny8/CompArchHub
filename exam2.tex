\documentclass[11pt]{article}
\usepackage{amsmath}
\usepackage{amssymb}
\usepackage{hyperref}
\usepackage{enumitem}
\usepackage{geometry}
\geometry{letterpaper, margin=1in}

\title{Computer Architecture \\ Sample Final Exam \\}
\author{}
\date{}

\begin{document}
\maketitle
\thispagestyle{empty}

\textbf{Instructions:}
\begin{itemize}[noitemsep]
    \item You have \textbf{180 minutes}. The exam consists of \textbf{10 questions}, worth \textbf{180 points total}.
    \item For multiple-choice questions, indicate your answer clearly (e.g., \(\bigcirc\)).
    \item For coding and math questions, \textbf{show all your work}.
    \item Assume a 64-bit RISC-V architecture unless otherwise stated.
\end{itemize}

\hrule
\vspace{0.2cm}

\textbf{Name: \hfill Student ID: \hfill Signature:} \\
\textit{``I pledge that I have neither given nor received unauthorized aid on this exam.''}

\vspace{0.2cm}
\hrule

\section*{Question 1: Number Systems (15 points)}
\begin{enumerate}[label=\textbf{Q1.\arabic*}]
    \item Convert \(0b11010101\) (8-bit signed) to decimal and hexadecimal. (2 points)
    \item Convert \(-75\) into a 16-bit two's complement binary number. If not possible, state why. (3 points)
    \item Represent \(0xC1A00000\) as an IEEE-754 single-precision floating-point number in the form \(x \times 2^y\), where \(1 \leq |x| < 2\). (5 points)
    \item Given \(X = 0b1101\) and \(Y = 0b1011\), compute \(X \oplus Y\) and \(X \& Y\). (5 points)
\end{enumerate}

\section*{Question 2: RISC-V Assembly (20 points)}
Write a RISC-V assembly function \texttt{reverse\_bits} that takes an integer in \texttt{a0}, reverses its bits, and returns the result in \texttt{a0}. For example, if \texttt{a0 = 0b1011}, the function should return \texttt{0b1101}. Assume a 32-bit integer. (20 points)

\section*{Question 3: Pipelining (20 points)}
A RISC-V pipeline executes the following sequence of instructions:
\begin{verbatim}
    add x5, x6, x7
    lw x8, 0(x5)
    sub x9, x8, x5
    beq x9, x0, label
\end{verbatim}
\begin{enumerate}[label=\textbf{Q3.\arabic*}]
    \item Identify all data hazards in the sequence. Specify the registers and instructions involved. (10 points)
    \item Explain how a forwarding unit would resolve one of these hazards. Include a diagram if necessary. (10 points)
\end{enumerate}

\section*{Question 4: Caches (20 points)}
Assume a 32-byte direct-mapped cache with 8-byte blocks and a 16-bit memory address.
\begin{enumerate}[label=\textbf{Q4.\arabic*}]
    \item Calculate the number of tag, index, and offset bits. (5 points)
    \item Determine whether each of the following addresses will result in a hit or miss if the cache is initially empty. The sequence of memory accesses is: \texttt{0x0010, 0x0018, 0x0020, 0x0010}. (10 points)
    \item Briefly discuss the advantages and disadvantages of direct-mapped caches compared to fully associative caches. (5 points)
\end{enumerate}

\section*{Question 5: Virtual Memory (20 points)}
Your system has a 64-bit virtual address space, 32 GiB of physical memory, and 4 KiB pages.
\begin{enumerate}[label=\textbf{Q5.\arabic*}]
    \item How many bits are required for the page offset, VPN, and PPN? (5 points)
    \item Compute the size of a single-level page table (in bytes). Assume each page table entry stores the PPN without additional metadata. (10 points)
    \item Explain how TLBs improve performance in a virtual memory system. (5 points)
\end{enumerate}

\section*{Question 6: Parallel Programming (20 points)}
The following code estimates \(\pi\) using the Monte Carlo method. It is parallelized with OpenMP:
\begin{verbatim}
    int count = 0;
    #pragma omp parallel for
    for (int i = 0; i < N; i++) {
        double x = random();
        double y = random();
        if (x * x + y * y <= 1) {
            #pragma omp critical
            count++;
        }
    }
\end{verbatim}
\begin{enumerate}[label=\textbf{Q6.\arabic*}]
    \item Explain why using \texttt{\#pragma omp critical} can reduce performance. Suggest an alternative. (10 points)
    \item Modify the code to avoid using \texttt{critical} while maintaining correctness. (10 points)
\end{enumerate}

\section*{Question 7: Control Logic (15 points)}
Given the following FSM:
\[
\begin{aligned}
\text{States: } & S_0, S_1, S_2, S_3 \\
\text{Transitions: } & S_0 \xrightarrow[]{0} S_0, \ S_0 \xrightarrow[]{1} S_1, \ S_1 \xrightarrow[]{0} S_2, \\
                    & S_1 \xrightarrow[]{1} S_3, \ S_2 \xrightarrow[]{0} S_0, \ S_2 \xrightarrow[]{1} S_3, \\
                    & S_3 \xrightarrow[]{0, 1} S_3
\end{aligned}
\]
Draw the state diagram and implement the FSM using logic gates. (15 points)

\section*{Question 8: Advanced Problem (25 points)}
You are given a CPU with:
- 4 stages: Fetch, Decode, Execute, Writeback
- A unified L1 cache with a hit rate of 80% and a latency of 3 cycles
- Main memory latency of 50 cycles

\begin{enumerate}[label=\textbf{Q8.\arabic*}]
    \item Calculate the effective CPI if the base CPI is 1.2, and 40% of instructions are memory accesses. (15 points)
    \item Discuss two architectural changes that could improve performance. (10 points)
\end{enumerate}

---

Good luck!
\end{document}
