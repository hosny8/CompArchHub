\documentclass[11pt]{article}
\usepackage{amsmath}
\usepackage{amssymb}
\usepackage{hyperref}
\usepackage{enumitem}
\usepackage{geometry}
\geometry{letterpaper, margin=1in}

\title{CS 61C: Computer Architecture \\
Fall 2024 \\
\textbf{Final Exam}}
\author{}
\date{}

\begin{document}
\maketitle
\thispagestyle{empty}

\textbf{Instructions:}
\begin{itemize}[noitemsep]
    \item You have \textbf{180 minutes}. The exam consists of \textbf{10 questions}, worth \textbf{100 points total}.
    \item For multiple-choice questions, clearly indicate your answer by filling the appropriate bubble.
    \item For coding questions, write your solutions clearly in the provided spaces.
    \item Assume RISC-V architecture for all assembly code questions unless stated otherwise.
\end{itemize}

\hrule
\vspace{0.2cm}

\textbf{Name: \hfill Student ID: \hfill Signature:} \\
\textit{``I pledge that I have neither given nor received unauthorized aid on this exam.''}

\vspace{0.2cm}
\hrule

\section*{Question 1: Number Representations (10 points)}
\begin{enumerate}[label=\textbf{Q1.\arabic*}]
    \item Convert the following 8-bit two's complement binary number to decimal: \texttt{0b11101100}. (2 points)
    \item Represent \(-45\) in 8-bit two's complement binary. If not possible, write \texttt{None}. (2 points)
    \item Interpret \texttt{0xC0800000} as an IEEE-754 single-precision floating-point number. Express your answer in the form \(x \times 2^y\), where \(1 \leq |x| < 2\). (3 points)
    \item True or False: All RISC-V instructions are 4 bytes in size. (1 point)
    \item If a cache has a hit rate of 95\%, a hit time of 2 cycles, and a miss penalty of 50 cycles, what is the Average Memory Access Time (AMAT)? (2 points)
\end{enumerate}

\section*{Question 2: RISC-V Assembly (12 points)}
Write a RISC-V assembly function that computes the maximum of an array of integers. The function should take two arguments:
\begin{itemize}[noitemsep]
    \item \texttt{a0}: the address of the array
    \item \texttt{a1}: the number of elements in the array
\end{itemize}
The result (maximum value) should be returned in \texttt{a0}. (12 points)

\textit{Assume all values are signed 32-bit integers. Use only the following registers: \texttt{a0, a1, t0, t1, t2}.}

\section*{Question 3: Pipelines (10 points)}
\begin{enumerate}[label=\textbf{Q3.\arabic*}]
    \item Identify the types of hazards in the following RISC-V program and specify the stages involved: (5 points)
    \begin{verbatim}
        addi t0, x0, 5
        addi t1, x0, 10
        mul t2, t0, t1
        add t3, t2, t1
    \end{verbatim}
    \item A RISC-V pipeline includes a forwarding unit to resolve hazards. Explain how this unit works and provide an example of a data hazard that it resolves. (5 points)
\end{enumerate}

\section*{Question 4: Caches (13 points)}
\begin{enumerate}[label=\textbf{Q4.\arabic*}]
    \item A cache has 64 lines, each of size 16 bytes, and uses direct mapping. How many bits are used for the offset, index, and tag in a 32-bit address? (3 points)
    \item Assume a program accesses the following sequence of memory addresses: 
    \[
    \texttt{0x200, 0x210, 0x220, 0x200, 0x230, 0x240, 0x200}
    \]
    Calculate the number of hits and misses if the cache is initially empty. (5 points)
    \item Explain the trade-offs between direct-mapped and fully associative caches. (5 points)
\end{enumerate}

\section*{Question 5: Virtual Memory (10 points)}
Suppose a system has a virtual address space of \(2^{32}\) bytes, a physical memory of \(2^{24}\) bytes, and a page size of \(2^{12}\) bytes.
\begin{enumerate}[label=\textbf{Q5.\arabic*}]
    \item How many entries are in the page table? (2 points)
    \item How many bits are required for the page offset, virtual page number (VPN), and physical page number (PPN)? (3 points)
    \item Discuss the advantages and disadvantages of using a Translation Lookaside Buffer (TLB). (5 points)
\end{enumerate}

\section*{Question 6: OpenMP Parallelization (10 points)}
\begin{enumerate}[label=\textbf{Q6.\arabic*}]
    \item Rewrite the following loop using OpenMP directives to parallelize it: (5 points)
    \begin{verbatim}
        for (int i = 0; i < N; i++) {
            array[i] = array[i] * 2;
        }
    \end{verbatim}
    \item Explain how OpenMP handles race conditions and how you can prevent them. (5 points)
\end{enumerate}

\section*{Question 7: Final Thoughts (1 point)}
If you could change one feature of the RISC-V ISA, what would it be and why? (1 point)

\end{document}
